\documentclass[a4paper,10pt]{article}
\usepackage[utf8]{inputenc}

%opening
\title{}
\author{}

\begin{document}

\maketitle

\begin{abstract}

\end{abstract}

\section{Descrição da Implementação}

O algoritmo de funcionamento do programa funciona com o auxílio das variáveis da Tabela \ref{tab:variaveis} onde também são descritas suas funções de controle. A Tabela \ref{tab:bits} contém os pinos do PIC utilizados como saídas e entradas.

Inicialmente, no \textit{loop} da rotina \textit{START}, o programa tem somente dois botões como entrada, \textit{BOTAO\_START} para iniciar e \textit{BOTAO\_N\_PS} para definir o número de jogadores da partida. Na seleção do número de jogadores, os bits \textit{BOT,2} e \textit{BOT,3} são setados ou limpados enquanto que a variável \textit{COUNTER} inicia como 0, 1 ou 2 para 4, 3 ou 2 jogadores, respectivamente. Isso se deve porque, logo que o contador atinge um valor igual a 3, o perdedor deve levar um choque.

Ao pressionar o botão de selecionar o número de jogadores, a rotina \textit{SET\_N\_PS} é chamada onde os bits \textit{ENABLED\_LED\_P3} e \textit{ENABLED\_LED\_P4} são lidos, caso tenham valor lógico alto, significa que os jogadores 3 e 4, respectivamente, estão habilitados e a rotina \textit{DISABLE\_P3P4} é chamada para desabilitá-los. Caso somente 2 estiverem habilitados, o jogador 3 é habilitado pela rotina \textit{SET\_P3} onde bit \textit{ENABLED\_LED\_P3} é setado para alto. Em último caso, se houverem 3 habilitados, a rotina \textit{SET\_P4} é chamada para setar como 1 o bit \textit{ENABLED\_LED\_P4}.

No começo de uma partida, o programa passa a operar no delay da rotina \textit{DELAY\_LED}. Sua função é manter \textit{START\_LED} setado até o momento de limpá-lo e os botões do jogo serem apertados, porém também checa constantemente, na sua ``subrotina'' \textit{DELAY2\_LED}, as entradas do botões do jogo para punir com um choque, através das rotinas \textit{SHOCK\_P1}, \textit{SHOCK\_P2}, \textit{SHOCK\_P3} e \textit{SHOCK\_P4}, o jogador que apertar o botão antes da hora certa e, em seguida. esperar o início de uma nova partida.

Quando o \textit{START\_LED} apagar, o programa estará rodando no loop da rotina \textit{MAIN}, definido como momento de decisão, onde os botões do jogo são novamente checados. Caso algum seja apertado, alguma das rotinas \textit{BOTAO\_P1\_PRESS}, \textit{BOTAO\_P2\_PRESS}, \textit{BOTAO\_P3\_PRESS} ou \textit{BOTAO\_P4\_PRESS} são chamadas para, além de incrementar 1 no valor de \textit{COUNTER}, setar o bit \textit{BOT,0} , \textit{BOT,1} , \textit{BOT,2} ou \textit{BOT,3} , respectivamente para cada jogador.

Durante o momento de decisão, os botões \textit{BOTAO\_P1}, \textit{BOTAO\_P2}, \textit{BOTAO\_P3} e \textit{BOTAO\_P4} passam a ser funcionais. No entanto, realmente funcionarão somente o botão daqueles jogadores que foram habilitados. Por exemplo, caso sejam somente dois oponentes, os botões \textit{BOTAO\_P3} e \textit{BOTAO\_P4} ainda funcionam como entrada, porém os seus estados de pressionado ou não já foram setados para o estado pressionado durante a selecão do número de jogadores, isto é, neste caso, \textit{BOT,2} e \textit{BOT,3} estão setados como 1 fazendo-os indeferentes neste momento. Ademais, a variável \textit{COUNTER} já foi incrementada em 1 para cada jogador desabilitado ou vice-versa para cada jogador é reabilitado. Os valores de \textit{BOT} e \textit{COUNTER} são armazenados em \textit{LAST\_COUNTER} e \textit{LAST\_BOT}. 

Ainda dentro do loop da \textit{MAIN}, existe uma rotina chamada \textit{CHECK\_COUNTER}. Nesta rotina, a variável \textit{COUNTER} é checada à procura de um valor igual a 3 para fazer a chamada da rotina \textit{FIND\_LOSER}.

A rotina \textit{FIND\_LOSER} procura pelo primeiro 0 entre os bits de \textit{BOT,0} , \textit{BOT,1} , \textit{BOT,2} e \textit{BOT,3} , nesta ordem, para setar algum dos bits de \textit{SHOCK\_LED\_P1}, {SHOCK\_LED\_P2}, {SHOCK\_LED\_P3} e {SHOCK\_LED\_P4} como 1 simbolizando o choque, respectivamente para o jogadores 1, 2, 3 e 4. Em sequência, chama a rotina \textit{DELAY\_SHOCK} que limita a duração do choque, limpa os bits \textit{SHOCK\_LED\_P1}, \textit{SHOCK\_LED\_P2}, \textit{SHOCK\_LED\_P3} e \textit{SHOCK\_LED\_P4}, carrega os valores de \textit{LAST\_COUNTER} e \textit{LAST\_BOT} em \textit{COUNTER} e \textit{BOT}, respectivamente, para recomeçar o jogo com a última configuração de jogadores.

\begin{table}
  \centering
  \caption{Variáveis de controle}
  \vspace{0.5cm}
  \label{tab:variaveis}
  \begin{tabular}{|p{3cm}|p{1.4cm}|p{6.6cm}|} \hline
    Variáveis 		& Endereco & Descrição 						\\ \hline
    COUNTER		& 0x20	   & Contador do número de botões apertados antes
				     de detectar quem perdeu a partida			\\ \hline
    LAST\_COUNTER	& 0x21	   & \textit{Backup} da configuração de CONT		\\ \hline
    BOT			& 0x22	   & Campo de 8 bits responsável por memorizar
				     o estado de apertado ou não durante o 
				     momento de decisão da partida			\\ \hline
    LAST\_BOT		& 0x23	   & \textit{Backup} da configuração de BOT 		\\ \hline  
    COUNT1		& 0x24	   & Contador 1 para o delay de duração do choque	\\ \hline
    COUNT2		& 0x25	   & Contador 2 para o delay de duração do choque	\\ \hline
    COUNT3		& 0x26	   & Contador 3 para o delay de duração do choque	\\ \hline
  \end{tabular}
\end{table}

\begin{table}
  \centering
  \caption{Bits de controle}
  \vspace{0.5cm}
  \label{tab:bits}
  \begin{tabular}{|p{3.4cm}|p{1.4cm}|p{6.2cm}|}\hline
    Bits 		& Pino		& Descrição						\\ \hline
    BOTAO\_P1		& PORTA,0	& Botão para mudar o estado do jogador 1 durante
					  o momento de decisão da partida			\\ \hline
    BOTAO\_P2		& PORTA,1	& Botão para mudar o estado do jogador 2 durante
					  o momento de decisão da partida			\\ \hline
    BOTAO\_P3		& PORTA,2	& Botão para mudar o estado do jogador 3 durante
					  o momento de decisão da partida			\\ \hline
    BOTAO\_P4		& PORTA,3	& Botão para mudar o estado do jogador 4 durante
					  o momento de decisão da partida			\\ \hline
    BOTAO\_START	& PORTA,4	& Botão para iniciar a partida				\\ \hline
    BOTAO\_N\_PS	& PORTA,5	& Botão para escolher o número de jogadores da partida	\\ \hline
    SHOCK\_LED\_P1	& PORTB,0	& LED que simboliza o sinal de choque no jogador 1	\\ \hline
    SHOCK\_LED\_P2	& PORTB,1	& LED que simboliza o sinal de choque no jogador 2	\\ \hline
    SHOCK\_LED\_P3	& PORTB,2	& LED que simboliza o sinal de choque no jogador 3	\\ \hline
    SHOCK\_LED\_P4	& PORTB,3	& LED que simboliza o sinal de choque no jogador 4	\\ \hline
    ENABLED\_LED\_P3	& PORTB,4	& LED indicador de participação do jogador 3 na partida	\\ \hline
    ENABLED\_LED\_P4	& PORTB,5	& LED indicador de participação do jogador 4 na partida	\\ \hline
    START\_LED		& PORTB,6	& LED que sinaliza o início do momento de decisão	\\ \hline
  \end{tabular}
\end{table}



\end{document}