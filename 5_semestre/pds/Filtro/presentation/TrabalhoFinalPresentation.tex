\documentclass{beamer}

% Setup appearance:

\usetheme{Darmstadt}
\usefonttheme[onlylarge]{structurebold}
\setbeamerfont*{frametitle}{size=\normalsize,series=\bfseries}
\setbeamertemplate{navigation symbols}{}


% Standard packages

\usepackage[brazil]{babel}
\usepackage[latin1]{inputenc}
\usepackage{times}
\usepackage[T1]{fontenc}
\usepackage[table]{xcolor}
 
% Setup TikZ

\usepackage{tikz}
\usetikzlibrary{arrows}
\tikzstyle{block}=[draw opacity=0.7,line width=1.4cm]

%diret\'orio das figuras
\graphicspath{../article}

\title[Filtro El\'iptico]{%
Filtro El\'iptico%
}

\author[Souza,Santos,Ara\'ujo]{
     Danilo~Souza\and
     Hugo~Santos\and
     Welton~Ara\'ujo
     }


\institute[Bel\'em]{
  \inst{1}%
  Universidade Federal do Par\'a
  }
\date[Bel\'em 2012]{
  03 de Julho de 2012
  }



\begin{document}

\begin{frame}
  \titlepage
\end{frame}

\begin{frame}{Agenda}
  \tableofcontents
\end{frame}

\section{Introdu\c{c}\~ao  Teste \s{c}}
\begin{frame}{Filtros Digitais}
	\begin{itemize}
	 \item	Filtros FIR
	 \item	Filtros IIR
		\begin{itemize}
		  \item Mapemamento de filtros anal\'ogicos
		  \item Menor ordem
		  \item Mais dif\'iceis de projetar
		  \item Problemas com Estabilidade
		\end{itemize}
	\end{itemize}
\end{frame}

\section{Filtro El\'iptico}

	\begin{frame}{Filtro El\'iptico}
		\begin{itemize}{Vantagens}
		  \item Maior declive na banda de transi\c{c}\~ao}
		  \item Menor ordem que outros filtros IIR
		\end{itemize}
		\begin{itemize}{Desvantagens}
		    \item N\~ao possui fase linear
			\begin{itemize}
			    \item Projetado somente em termos de magnitude
			\end{itemize}
		\end{itemize}
	\end{frame}

\subsection{Projeto de Filtros El\'ipticos}
   
	\begin{frame}{Abordagens}
		\begin{itemize}
			\item Duas abordagens
		\end{itemize}
		\begin{itemize}
			\item  Abordagem I
				\begin{itemize}
					\item Projetar filtro Passa-Baixa anal\'ogico
					\item Realizar transforma\c{c}\~ao em frequ\^encia (s \(\rightarrow\) s)
					\item Aplicar transforma\c{c}\~ao do filtro (s \(\rightarrow\) z)
				\end{itemize}
	\end{itemize}
		\begin{itemize}
			\item Abordagem II
				\begin{itemize}
					\item Projetar filtro Passa-Baixa anal\'ogico
					\item Aplicar transforma\c{c}\~ao do filtro (s \(\rightarrow\) z)
					\item Realizar transforma\c{c}\~ao em frequ\^encia (z \(\rightarrow\) z)
				\end{itemize}
		\end{itemize}
	\end{frame}

	\begin{frame}{Projeto dos filtros}
		\begin{itemize}
		 \item Encontrar a frequ\^encia digital \omega
		\end{itemize}
		\begin{itemize}
		 \item Encontrar a frequ\^encia distorcida \Omega
		\end{itemize}
		\begin{itemize}
		 \item \'E preciso encontrar algumas constantes para calcular a ordem do filtro
		\end{itemize}
	\end{frame}


\subsection{A Simula\c{c}\~ao}


  \end{frame}

  

\end{document}
