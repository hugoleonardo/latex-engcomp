\documentclass{beamer}

% Setup appearance:

\usetheme{Darmstadt}
\usefonttheme[onlylarge]{structurebold}
\setbeamerfont*{frametitle}{size=\normalsize,series=\bfseries}
\setbeamertemplate{navigation symbols}{}


% Standard packages

\usepackage[brazil]{babel}
\usepackage[latin1]{inputenc}
\usepackage{times}
\usepackage[T1]{fontenc}
\usepackage{amsmath}% http://ctan.org/pkg/amsmath
%\usepackage[table]{xcolor}
\usepackage{multicol}
\usepackage{textcomp} 

% Setup TikZ
\usepackage{tikz}
\usetikzlibrary{arrows}
\tikzstyle{block}=[draw opacity=0.7,line width=1.4cm]

%diretório das figuras
\graphicspath{../article}

\title[Utiliza��o de Redes Neurais para Ger�ncia de Servidores 
Virtuais Web]{%
Utiliza��o de Redes Neurais para Ger�ncia de Servidores 
Virtuais Web%
}

\author[Souza,Medeiros]{
     Danilo~Souza$^{1}$ \and
     Iago~Medeiros$^{1}$
     }


\institute[Bel�m]{
  \inst{1}%
  Universidade Federal do Par�
  }
\date[Bel�m 2012]{
  20 de Junho de 2013
  }



\begin{document}

\begin{frame}
  \titlepage
\end{frame}

\begin{frame}
  \tableofcontents
\end{frame}


\section{Introdu��o}
	
	\begin{frame}{Arquitetura do iago}
	
	\end{frame}

\subsection{Introdu��o Geral}

	\begin{frame}{Introdu��o do Artigo}
		\begin{itemize}
			\item Alto consumo de energia em \textit{datacenters} (40\% para equipamentos e 60\% para infra-estrutura)
			\item Virtualizar servidores reduz o consumo de energia
				\begin{itemize}
					\item Gera maior ociosidade 			
				\end{itemize}
			\item Normalmente os equipamentos s�o superdimensionados (ociosidade)
			\item O autor prop�e uma nova pol�tica de ger�ncia de servidores \textit{web}
		\end{itemize}		
	\end{frame}

\subsection{Introdu��o � Virtualiza��o}

	\begin{frame}
	teste
	\end{frame}

	\begin{frame}{Arquitetura do Xen}
	
	\end{frame}

\subsection{Introdu��o � Redes Neurais Artificias (RNA)}

	\begin{frame}
	
	\end{frame}

\section{Trabalhos Relacionados}

	\begin{frame}
	
	\end{frame}

\subsection{Ger�ncia de VM's}

	\begin{frame}
	
	\end{frame}

\subsection{Balanceamento de Carga utilizando RNA}

	\begin{frame}
	
	\end{frame}

\section{Modelo de RNA utilizado}

	\begin{frame}
	
	\end{frame}

\subsection{Mapas Auto-Organiz�veis}

	\begin{frame}
	
	\end{frame}

\section{Arquitetura Proposta}

	\begin{frame}
	
	\end{frame}

\subsection{Par�metros Analisados}	

	\begin{frame}
	
	\end{frame}

\subsection{Interven��es da API}

	\begin{frame}
	
	\end{frame}

\subsection{Pol�ticas de Reconfigura��o}

	\begin{frame}
	
	\end{frame}

\section{Testes}

	\begin{frame}
	
	\end{frame}

\subsection{Avalia��o}

	\begin{frame}
	
	\end{frame}

\subsection{Workload utilizando site da NASA}

	\begin{frame}
	
	\end{frame}

\section{Conclus�o e Trabalhos futuros}

	\begin{frame}
	
	\end{frame}









\end{document}