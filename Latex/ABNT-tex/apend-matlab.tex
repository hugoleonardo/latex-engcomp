\chapter{C�digos MATLAB}
\label{apend-cod-MATLAB}

\begin{mylisting}
\begin{verbatim}
%%%%%%%%%%%%%%%%%%%%%%%%%%%%%%%%%%%%%%%%%%%%%%%%%%%%%%%%%
%                                                       %
%  Alexandre B. de Lima                                 %
%                                                       %
%   04/11/2007                                          %
%%%%%%%%%%%%%%%%%%%%%%%%%%%%%%%%%%%%%%%%%%%%%%%%%%%%%%%%%
%
% function emse() 
%
% Given a time series FARIMA(p,d,q), an estimate of the long memory
% parameter d (= H-1/2), and an adjusted AR(p2), this function returns: 
%   1) "k-steps-ahead-forecasts"  based on the last 100 observations
%      (k=1,...,k_max=100) given by the AR(p2) and the TARFIMA(L) models.
%   2) an Excel Table with the Empirical Mean-Square Errors (EMSE) for
%      several values of k_max (k_max= 7,10,20,50,100)
%
% Note that TARFIMA(L) is a finite-dimensional (i.e.,truncated) 
% state-space representation of an FARIMA(p,d,0) model. For more details, 
% please refer to the paper  
% "State-Space Modeling of Long-Range Dependent Teletraffic", ITC 2007,
% LNCS 4516, pp.260-271, 2007 (20th Int'l Teletraffic Congress, Ottawa, 
% Canada).
%
% inputs: 
%          timeseries = name of the text file which contains the time 
%                       series under analysis (ASCII column vector)
%          d          = estimated value of "d" for "timeseries" (I use 
%                       the estimate given by S+FinMetrics) 
%          name_ar_farima = name of a text file with the estimated AR coef.
%                       of FARIMA (if applicable)
%          name_ma_farima = name of a text file with the estimated MA coef.
%                       of FARIMA (if applicable)
%          L          = order of the finite-dimensional representation
%          namear     = name of a text file which contains the parameters 
%                       of the AR fitted model 
%
% outputs:
%          Table with EMSEs, Figures with forecasts 
%
% Copyright(c) 2007 Universidade de S�o Paulo, Laborat�rio de Comunica��es
% e Sinais
% 
% Permission is granted for use and non-profit distribution providing that 
% this notice be clearly maintained. The right to distribute any portion 
% for profit or as part of any commercial product is specifically reserved
% for the author.

%function emse() 

clear
close all

% note that : a) vectors with coefficients are column vectors, and
%             b) vetors with observations are row vectors

% ================================================================
% INPUT DATA

%  Prompt for time series
timeseries = input('file with the time series: ','s');

%  Prompt for estimated FARIMA model
d = input('parameter "d": ');

yes = 'y';
yes2 = 'Y';
j1 = input('AR coefficients? Y/N [Y]: ','s');
if ( strcmp(yes,j1) | strcmp(yes2,j1) ) 
    name_ar_farima = input('file with the estimated AR coef. of FARIMA:
     ','s');
    ar_farima = load(name_ar_farima);
    ar_farima = -1*ar_farima; 
% remember that the coefficients given by S-PLUS are the 
% negative of the coefficients used in MATLAB (convention)
% See S-PLUS Guide to Statistics, Vol. 2 
    ar_farima = [1; ar_farima]; % MATLAB convention
else
    ar_farima = [1];
end;

j2=input('MA coefficients? Y/N [Y]: ','s');
if ( strcmp(yes,j2) | strcmp(yes2,j2) )
    name_ma_farima = input('file with the estimated MA coef. of FARIMA:
     ','s');
    ma_farima = load(name_ma_farima);
    ma_farima = -1*ma_farima; 
% remember that the coefficients given by S-PLUS are the 
% negative of the coefficients used in MATLAB (convention)
    ma_farima = [1; ma_farima]; % MATLAB convention
else 
    ma_farima = [1];
end;

L = input('order of the finite-dimensional representation: ');
namear = input('file with the coef. of the estimated AR model: ','s');

fullseries =load(timeseries)';
N = length(fullseries);
%mu = mean(fullseries);

% =================================
% Prediction from T-FARIMA(L) model

for j = 0:100 
    D(j+1) = -gamma(j-d)/(gamma(j+1)*gamma(-d));
end

D = -1*D; % D(B) = (1-B)^d = 1 -d_1B -d_2B^2 - ...
% "D" is a row vector

phi_temp = conv(ar_farima,D');

% Truncated AR(L) polynomial (1-phi_1B-phi_2B^2 - ... - phi_LB^L)
phi_B = ldiv(phi_temp,ma_farima,(L+1)); 
phi_B2 = -1*phi_B(2:(L+1)); %coef. of AR(L) (S+ convention) 
                            %column vector
[m,n]=size(phi_B2);
 
if (m<n)
    phi_B2 = phi_B2';
end

k = [7 10 20 50 100];
% k denotes the number of steps to predict ahead
% k_max=100

% ma100 = ldiv(ma_farima,phi_temp,101); % ldiv(b,a,N) is a function for 
                                        % the power series expansion of 
                                        % polynomial phi_B
% Now we must invert and truncate the AR(L) representation
% Yields the same result of: ma100 = ldiv(ma_farima,phi_temp,101)
ma100 = ldiv(1,phi_B,101);
ma100 = ma100(2:length(ma100));
% Vector with the coef. of the MA(100) representation of T-FARIMA(L) 
% [psi1 psi2 ... psi100], calculated by S-PLUS
% do not change signs! (as stated in section "DETAILS" of the "arma2ma()"
%                       function help )

%prev = zeros(length(fullseries)); % initialize row vector with predictions 

ma100quad = ma100.^2; % coefficiente to the square
var_erro = zeros(1,k(1)); % initialize row vector with the variances 
                          % of the k-step ahead prediction errors

%t = 2700;
%t=866;
% origin of prediction
                          
for j=1:length(k)

    t=length(fullseries)-k(j); 
    % origin of prediction
    
    % estimate mean of time series
    mu = mean(fullseries(t:-1:1));  
    
    % Initialize row vector Z (with L previous observations)
    Z  = fullseries(t:-1:t-(length(phi_B2)-1)) - mu;
    % Z = [Z_358 Z_357 ... Z_309]

    % column vector I of innovations
    I = filter(phi_B,1,(fullseries'-mu));
    pot_inov = var(I);

    for h=1:k(j)    
        prev(h) = Z*phi_B2;
        Z = [prev(h) Z(1:(length(phi_B2)-1))];
    end    

    prev = prev + mu; % predictions
    x = fullseries(t+1:t+k(j)); % future observations
    EQMPt(j) = sum((prev-x).^2)/k(j);
    erro = x-prev;
    
    if j == length(k)
        for h=1:k(j)    
            if h==1
                var_erro(h) = 1*pot_inov; % vide Eq.(9.11) Morettin(2004)
            else
                var_erro(h) = (ma100quad(h-1)*pot_inov) + var_erro(h-1);
            end 
        end   
    end   
end

upper_bound = prev + 1.96*sqrt(var_erro);
lower_bound = prev - 1.96*sqrt(var_erro);

erroabs = abs(erro);

% ===========================

% Prediction from AR(p) model 

ar = load(namear);
p = length(ar);
ar = -1*ar; % remember that the coefficients given by S-PLUS are the 
            % negative of the coefficients used in MATLAB (convention)
            
ar = [1; ar]; % MATLAB convention
ARorder = strcat('AR(',num2str(p),')')
B = [1];
[z_ar,p_ar,G] = tf2zpk(B,ar);
zplane(z_ar,p_ar,'k')
title(['Diagrama de P�los e Zeros do modelo AR(',int2str(p),')'])
figure

ar2 = -1*ar(2:length(ar)); % S-PLUS convention

ar_ma100 = ldiv(1,ar,101);
ar_ma100 = ar_ma100(2:length(ar_ma100))';
% Vector with the coef. of the MA(100) representation of AR(p) 
% [psi1 psi2 ... psi100], calculated by S-PLUS
% do not change signs! (as stated in section "DETAILS" of the "arma2ma()"
%                       function help )

%prev2 = zeros(length(fullseries)); 
% initialize row vector with predictions 
ar_ma100quad = ar_ma100.^2; % coef. to the square
var_erro2 = zeros(1,k(1));  
% initialize row vector with the variances 
% of the k-step ahead prediction errors


for j=1:length(k)
    
    t=length(fullseries)-k(j); % origin of prediction
    
    % estimate mean of time series
    mu = mean(fullseries(t:-1:1));  
    
    % Initialize row vector Z2 (p observations)
    Z2 = fullseries(t:-1:t-(length(ar2)-1))- mu;

    % column vector I2 of innovations
    I2 = filter(ar,1,(fullseries'-mu));
    pot_inov2 = var(I2);
    
    for h=1:1:k(j)    
        prev2(h) = Z2*ar2;
        Z2 = [prev2(h) Z2(1:(length(ar2)-1))];
    end

    prev2 = prev2 + mu;
    x2 = fullseries(t+1:t+k(j));
    EQMPt2(j) = sum((prev2-x2).^2)/k(j);  
    erro2 = x2-prev2;

    if j == length(k)
        for h=1:k(j)    
            if h==1
                var_erro2(h) = 1*pot_inov2; 
                % vide Eq.(9.11) Morettin(2004)
            else
                var_erro2(h) = (ar_ma100quad(h-1)*pot_inov2) + 
                var_erro2(h-1);
            end 
        end   
    end   
end

upper_bound2 = prev2 + 1.96*sqrt(var_erro2);
lower_bound2 = prev2 - 1.96*sqrt(var_erro2);

erro2abs = abs(erro2);

delta_error_abs = erro2abs - erroabs ;
% =====================================
%ARorder = strcat('AR(',num2str(p),')')
TFARIMAorder = strcat('TARFIMA(',num2str(L),')')

EQM = [EQMPt2',EQMPt'];

emse1 = {'k',ARorder,TFARIMAorder; k(1),EQM(1,1),EQM(1,2); k(2),EQM(2,1),
EQM(2,2); k(3),EQM(3,1),EQM(3,2);k(4),EQM(4,1),EQM(4,2); k(5),EQM(5,1),
EQM(5,2)}

xlswrite('emse.xls', emse1, 'teste');

%==============
% Draw graphics

% Estimation of Power spectral density (PSD) using Welch's method
% Advantages: simultaneous reduction of a) bias due to leakage (via data 
% tapering) and b) variability (by averaging the periodograms of 
% Nb blocks of size Ns) of the periodogram.

[Pxx,w] = pwelch(fullseries);
plot(w/(2*pi),10*log10(Pxx))
%xlabel('normalized frequency (cycle/sample)')
xlabel('f (ciclo/amostra)')
%ylabel('power/frequency (dB/cycle/sample)')
ylabel('dB/ciclo/amostra')
%title('PSD via WOSA')
title('DEP estimada (m�todo WOSA)')

figure
[Pxx,w] = pwelch(I);
plot(w/(2*pi),10*log10(Pxx))
xlabel('f (ciclo/amostra)')
ylabel('dB/ciclo/amostra')
title('DEP estimada (m�todo WOSA) dos res�duos do modelo TARFIMA')

t=length(fullseries)-max(k); % origin of prediction using k_max=100
                             % t=1106 for the Nile River minima series
                             % used by Beran in the first chapter of
                             % his book
%n = (t-max(k)+1):1:N;
n = (t-max(k)+1):1:(t+max(k));

figure
plot(n,fullseries(n),'k.-')
hold on
%pause
%plot((t+1:N),prev,'.:g')
plot((t+1:t+max(k)),prev,'.-g')
hold on
%pause
%plot((t+1:N),upper_bound,'x:g')
plot((t+1:t+max(k)),upper_bound,'x:g')
hold on
%pause
%plot((t+1:N),lower_bound,'*:g')
plot((t+1:t+max(k)),lower_bound,'*:g')
hold on
%pause
%plot((t+1:N),prev2,'.:r')
plot((t+1:t+max(k)),prev2,'.:r')
hold on
%pause
%plot((t+1:N),upper_bound2,'x:r')
plot((t+1:t+max(k)),upper_bound2,'x:r')
hold on
%pause
%plot((t+1:N),lower_bound2,'*:r')
plot((t+1:t+max(k)),lower_bound2,'*:r')

figure
plot((t+1:t+max(k)),delta_error_abs,'k.-')
xlabel('h')
%ylabel('difference of absolute values of prediction errors')
ylabel('diferen�a entre os erros absolutos de previs�o')

%==========================================
% Comparison of TARFIMAs with L=10, and 100

L2 = 10;
phi_B3 = ldiv(phi_temp,ma_farima,(L2+1)); 
phi_B4 = -1*phi_B3(2:(L2+1)); 
[m,n]=size(phi_B4);
 
if (m<n)
    phi_B4 = phi_B4';
end

ma100L2 = ldiv(1,phi_B3,101);
ma100L2 = ma100L2(2:length(ma100L2));

ma100quadL2 = ma100L2.^2; 
var_erro3 = zeros(1,max(k)); 

mu = mean(fullseries(t:-1:1));  
Z3  = fullseries(t:-1:t-(length(phi_B4)-1)) - mu;
I3 = filter(phi_B3,1,(fullseries'-mu));
pot_inov3 = var(I3);

for h=1:max(k)    
    prev3(h) = Z3*phi_B4;
    Z3 = [prev3(h) Z3(1:(length(phi_B4)-1))];
end    

prev3 = prev3 + mu; % predictions
x = fullseries(t+1:t+max(k)); % future observations
    
erro3 = x-prev3;
    
for h=1:max(k)    
    if h==1
       var_erro3(h) = 1*pot_inov3; % vide Eq.(9.11) Morettin(2004)
    else
       var_erro3(h) = (ma100quadL2(h-1)*pot_inov2) + var_erro3(h-1);
    end 
end   
                          
upper_bound3 = prev3 + 1.96*sqrt(var_erro3);
lower_bound3 = prev3 - 1.96*sqrt(var_erro3);         

n = (t-max(k)+1):1:(t+max(k));

figure
plot(n,fullseries(n),'k.-')
hold on
%pause
%plot((t+1:N),prev,'.:g')
plot((t+1:t+max(k)),prev,'.:g')
hold on
%pause
%plot((t+1:N),upper_bound,'x:g')
plot((t+1:t+max(k)),upper_bound,'x:g')
hold on
%pause
%plot((t+1:N),lower_bound,'*:g')
plot((t+1:t+max(k)),lower_bound,'*:g')
hold on
%pause
%plot((t+1:N),prev2,'.:r')
plot((t+1:t+max(k)),prev3,'.:b')
hold on
%pause
%plot((t+1:N),upper_bound2,'x:r')
plot((t+1:t+max(k)),upper_bound3,'x:b')
hold on
%pause
%plot((t+1:N),lower_bound2,'*:r')
plot((t+1:t+max(k)),lower_bound3,'*:b')
\end{verbatim}
\end{mylisting}

\clearpage
\begin{mylisting}
\begin{verbatim}
%%%%%%%%%%%%%%%%%%%%%%%%%%%%%%%%%%%%%%%%%%%%%%%%%%%%%%%%%
%                                                       %
%  Alexandre B. de Lima                                 %
%                                                       %
%   14/10/2007                                          %
%%%%%%%%%%%%%%%%%%%%%%%%%%%%%%%%%%%%%%%%%%%%%%%%%%%%%%%%%
%
% This script gives measures of the k-step forecastability of a time series
% as defined in p.24 of "An Introduction to Long-Memory Time Series Models 
% and Fractional Differencing", Granger and Joyeux, 1980. 
% 
% Copyright(c) 2007 Universidade de S�o Paulo, Laborat�rio de Comunica��es
% e Sinais
% 
% Permission is granted for use and non-profit distribution providing that 
% this notice be clearly maintained. The right to distribute any portion 
% for profit or as part of any commercial product is specifically reserved
% for the author.

clear
close all

% note that : a) vectors with coefficients are column vectors, and
%             b) vetors with observations are row vectors

% =====================================================================

% Forecasting potential of the long-memory models (predictions based on 
% an infinite past)

ar_farima = [1];
ma_farima = [1];
d = input('parameter "d": '); % parameter "d"

M = 200;  % to derive the first M coef. of the MA(inf) representation

for j = 0:M 
    D(j+1) = -gamma(j-d)/(gamma(j+1)*gamma(-d)); 
    % Eq.(16.30), p�g. 475, "An�lise de S�ries Temporais", 
    % Morettin & Toloi,2004
end

D = -1*D; % D(B) = (1-B)^d = 1 -d_1B -d_2B^2 - ... (estimated truncated 
          % fractional difference filter)
          % "D" is a row vector 

phi_B = conv(ar_farima,D'); % AR polynomial (1-phi_1B-phi_2B^2 - ...) 
phi_B2 = -1*phi_B(2:length(phi_B)); 
%coef. of AR polynomial (S+ convention) 

ma100 = ldiv(ma_farima,phi_B,M); % ldiv(b,a,N) is a function for 
                                   % the power series expansion of 
                                   % polynomial phi_B
ma100 = ma100';
ma100quad = ma100.^2; % coefficients to the square

Sk = cumsum(ma100quad); % vector with the variances of the k-step forecast 
                        % errors (length = 100) for the ARFIMA model

% let x_t be an FARIMA(0,d,0) of the form 
%       (1-B)^d{x_t} = w_t
% where it is assumed that var{w_t}=1 for convenience.
% The variance of x_t is given by 
%       var{x_t} = gamma(-2d+1)/gamma^2(-d+1)
% (see, Eq.(3.2) in paper "Fractional Differencing", Hosking, 1981) 
Vd = gamma(-2*d+1)/(gamma(-d+1))^2 ;                       

k = [1 10 20 100]; % so that you can reproduce the results presented in 
                   % Tables 8.7 and 8.8 by Jan Beran in "Statistics for 
                   % Long-Memory Processes", Chapman & Hall, 1994

% quantity "Rk" measures the k-step ahead forecasting potential of the LRD
% model, i.e., it's a standardized measure of how well one can predict 
% x_{t+k} given the infinite past

for j=1:length(k)
    Rk(j) = 1 - Sk(k(j))/Vd; 
end

%========================================================================
% Forecasting potential of the TARFIMA(L), whose predictions are based on
% L previous observations

% INPUT DATA

L = input('order of the finite-dimensional representation: ');
%d2 = input('estimate of "d": ');

for j = 0:L 
    D2(j+1) = -gamma(j-d)/(gamma(j+1)*gamma(-d));
end

D2 = -1*D2; 

phi2_B = conv(ar_farima,D2'); % AR polynomial (1-phi_1.B-phi_2.B^2 - ...) 
phi2_B2 = -1*phi2_B(2:length(phi2_B)); 
%coef. of AR polynomial (S+ convention) 

ma100_2 = ldiv(ma_farima,phi2_B,M); 
ma100_2 = ma100_2';
ma100quad_2 = ma100_2.^2;                                    
                                    
Sk2 = cumsum(ma100quad_2); 
% vector with the variances of the k-step forecast errors (length = 100)
% for the TARFIMA model  

% let x_t be approximated by an TARFIMA(L) model as 
%       (1 - phi_1.B - phi_2.B^2 - ... - phi_L.B^L)x_t = w_t
% where it is assumed that var{w_t}=1 for convenience.
% x_t has the MA(inf) representation
%       x_t = (1 - phi_1.B - phi_2.B^2 - ... - phi_L.B^L)^{-1}w_t  or
%       x_t = (1 + psi_1.B + psi_2.B^2 + ... )w_t
% and the variance of x_t is given by 
%       var{x_t} = 1 + sum_{j=1}^{inf}(psi_j^2)
% (see, Eq.(3.3.1) in p.91 of "Time Series: Theory and Methods", 2nd Ed., 
%  Brockwell and Davis, 1991)
% Below, Sk2(M) gives an estimate of var{x_t}
Vd2 =  Sk2(M);  

% quantity "R2k" measures the k-step ahead forecasting potential of the 
% truncated FARIMA model, i.e., it's a standardized measure of how well one 
% can predict x_{t+k} given "L" (=order of TARFIMA model) past observations
for j=1:length(k)
    R2k(j) = 1 - Sk2(k(j))/Vd2; 
end

% ========================================================================

% Forecasting potential for the AR(1) process with with rho(1)=1/9 
% and rho(1)=2/3. The lag-1 correlation is chosen such that it is the same
% as for a ARFIMA(0,d,0) process with H=0.6 and 0.9 respectively.
rho_1 = 1/9;
rho_2 = 2/3;

% let x_t be an AR(1) of the form 
%       (1-phi_1)^d{x_t} = w_t
% It is assumed that var{w_t}=1 for convenience.

Vh06_AR1 = 1/(1 -rho_1^2); % variance of AR(1) process with H=0.6;
Vh09_AR1 = 1/(1 -rho_2^2); % variance of AR(1) process with H=0.9;
                           % see Brockwell & Davis (1991), 
                           % Morettin & Toloi (2004)

for j=1:100                           
    Skh06_AR1(j) = Vh06_AR1*(1 - rho_1^(2*j));
    Skh09_AR1(j) = Vh09_AR1*(1 - rho_2^(2*j));
end

for j=1:length(k)
    R3k(j) = 1 - Skh06_AR1(k(j))/Vh06_AR1; 
    R4k(j) = 1 - Skh09_AR1(k(j))/Vh09_AR1;
end
\end{verbatim}
\end{mylisting}

\clearpage
\begin{mylisting}
\begin{verbatim}
%%%%%%%%%%%%%%%%%%%%%%%%%%%%%%%%%%%%%%%%%%%%%%%%%%%%%%%%%
%                                                       %
%  Alexandre B. de Lima                                 %
%                                                       %
%   28/10/2007                                          %
%%%%%%%%%%%%%%%%%%%%%%%%%%%%%%%%%%%%%%%%%%%%%%%%%%%%%%%%%

clear
close all

j=sqrt(-1);

d = input('parameter "d": ');
%d=0.3859;
H = d + 0.5;
da = 1; %potencia do ruido branco Gaussiano

yes = 'y';
yes2 = 'Y';
j1 = input('AR coefficients? Y/N [Y]: ','s');
if ( strcmp(yes,j1) | strcmp(yes2,j1) ) 
    name_ar_farima = input('file with the estimated AR coef. of FARIMA:
     ','s');
    ar_farima = load(name_ar_farima);
else
    ar_farima = [1];
end;

j2=input('MA coefficients? Y/N [Y]: ','s');
if ( strcmp(yes,j2) | strcmp(yes2,j2) )
    name_ma_farima = input('file with the estimated MA coef. of FARIMA:
     ','s');
    ma_farima = load(name_ma_farima);
else 
    ma_farima = [1];
end;

ar_farima = -1*ar_farima; 
% remember that the coefficients given by S-PLUS are the 
% negative of the coefficients used in MATLAB (convention)
% See S-PLUS Guide to Statistics, Vol. 2 

ar_farima = [1; ar_farima];
ma_farima = -1*ma_farima;                          
ma_farima = [1; ma_farima];

N=2048;
Deltaf = (0.5/N);

i=0;

% Calculo da Densidade Espectral de Potencia do modelo FARIMA(p,d,0) 
for f=(0.5/N):(0.5/N):0.5
    A = [1 -exp(-j*2*pi*f)];
    B = (abs(sum(A))^(-2*d)*(da^2));
    if (length(ma_farima)>1)
       A2 =  
    end
    C = [1 -0.3014*exp(-j*2*pi*f) 0.3988*exp(-j*2*pi*f*2)];    
    D = (abs(sum(C)))^2;
    i = i+1; 
    
    Sd(i) = B/D;
end

Pdf = Deltaf*Sd;
Pd  = sum(Pdf);

% Estima��o da DEP da seria simulada
se =dlmread('fGn_(4096)pts_alpha(0.8)_NrOfVM(1)_
TopLev(9)_realiz3_filtroIIR2.txt');
[Pxx,w] = pwelch(se,[],[],(2*N));

Pxx = Pxx(2:N+1);

Pf = Deltaf*Pxx;
P  = sum(Pf);

% Calculo da DEP do modelo T-FARIMA  
ar_farima = [0.3014; -0.3988]; 
% sinais dos coeficientes de acordo com a conven��o do S-PLUS = -1*Matlab

% Determina Polinomio "Vphi(B)=phi(B)D(B)" de acordo com a 
% Eq. (16.26), pg. 475, do Livro "An�lise de S�ries Temporais", de 
% Morettin e Toloi, 
% onde D(B) = (1-B)^d = 1 -d(1)B -d(2)B^2 - ...

ar_farima = -1*ar_farima; 
ar_farima = [1; ar_farima];

L=50;
M = length(ar_farima);
K = L-(M-1);

% calcula d(j)
for j = 0:K 
    D(j+1) = -gamma(j-d)/(gamma(j+1)*gamma(-d));
end

D = -1*D; % "D" � um vetor linha

Vphi_B = conv(ar_farima,D');
Vphi_B2 = -1*Vphi_B(2:length(Vphi_B)); 

Btfarima = 1;
Atfarima = Vphi_B;
F=(0.5/N):(0.5/N):0.5;
H = freqz(Btfarima,Atfarima,F,1);
St = abs(H).^2;

Ptf = Deltaf*St;
Pt  = sum(Ptf);

% Calculo da DEP do modelo AR estimado 
Bar = 1;
Aar = load('ar10h09.txt');
Aar = -1*Aar; 
Aar = [1; Aar];
H2 = freqz(Bar,Aar,F,1);
Sar = abs(H2).^2;

Par = Deltaf*Sar;
Ptar  = sum(Par);

R=Pt/Pd;
R1=Pt/P;
R2=Pt/Ptar;

f=(0.5/N):(0.5/N):0.5;

figure %figura 1
plot(f,10*log10(St),'r')
xlabel('frequency')
ylabel('PSD (dB)')

hold on
pause

plot(f,10*log10(R*Sd),'g')

hold on
pause

plot(f,10*log10(R2*Sar),'g')

erro = 10*log10(St)-10*log10(R*Sd);

figure %figura 2
plot(f,10*log10(St)-10*log10(R*Sd))
xlabel('frequency')
ylabel('dB')

figure %figura 3
loglog(f,R*Sd,'g')
xlabel('frequency')
ylabel('PSD')

hold on
pause
loglog(f,St,'r')

hold on
pause
loglog(f,R1*Pxx,'m')

%hold on
%pause
%loglog(f,R2*Sar,'b')
\end{verbatim}
\end{mylisting}

\clearpage
\begin{mylisting}
\begin{verbatim}
%%%%%%%%%%%%%%%%%%%%%%%%%%%%%%%%%%%%%%%%%%%%%%%%%%%%%%%%%
%                                                       %
%   Generate.m                                          %
%                                                       %
%   Fernando L. de Mello                                %
%   Alexandre B. de Lima                                %
%                                                       %
%   10/2006                                             %
%   12/2007                                             %
%                                                       %
%%%%%%%%%%%%%%%%%%%%%%%%%%%%%%%%%%%%%%%%%%%%%%%%%%%%%%%%%
%   References:
%    1) J.-A. B�ckar, "A Framework for Implementing Fractal Traffic Models 
%    in Real Time", M.Sc. Thesis, SERC, Melbourne,2000.
%    2) F. L. de Mello, "Estudo e Implementa��o de um Gerador de Tr�fego 
%    com Depend�ncia de Longa Dura��o", M.Sc. Thesis, Univ. of Sao Paulo, 2006.   
%    3) F. L. de Mello, A. B. de Lima, M. Lipas, J. R. A. Amazonas, 
%    "Gera��o de S�ries Auto-Similares Gaussianas via Wavelets para Uso em 
%    Simula��es de Tr�fego", IEEE Latin America Transactions, Mar�o, 2007.
%    4) A. B. de Lima, F. L. de Mello, M. Lipas, J. R. A. Amazonas,
%    "A Generator of Teletraffic with Long and Short-Range Dependence",
%    12th CAMAD Workshop (part of the 18th IEEE PIMRC07), Greece, 2007.
%    5) D. B. Percival and A. T. Walden, "Wavelet Methods for Time Series 
%    Analysis", Cambridge University Press, 2000.
%    6) R. H. Riedi, M. S. Crouse, V. J. Ribeiro, R. G. Baraniuk, 
%    "A multifractal wavelet model with application to network traffic"
%    IEEE Transactions on Information Theory, 45(3), pp.992-1018, April, 1999.
%%%%%%%%%%%%%%%%%%%%%%%%%%%%%%%%%%%%%%%%%%%%%%%%%%%%%%%%%%%%%%%%%%%%%%%%%%%%%%% 
%
%    'Generate.m' is a wavelet-based generator of self-similar sample paths of 
%    Fractional Gaussian Noise (FGN) and Multifractal Wavelet Model (MWM). 
%    It works in conjunction with functions 'Model.m' and 'Recon.m'
%
%%%%%%%%%%%%%%%%%%%%%%%%%%%%%%%%%%%%%%%%%%%%%%%%%%%%%%%%%%%%%%%%%%%%%%%%%%%%%%%
% Models:
% 1 - wavelet coeficients as white noise random processes (FGN synthesis)
% 2 - MWM
% 3 - wavelet coeficients as AR(1) random processes (FGN synthesis)
% 9 - Reconstrucao Analise Matlab
%%%%%%%%%%%%%%%%%%%%%%%%%%%%%%%%%%%%%%%%%%%%%%%%%%%%%%%%%%%%%%%%%%%%%%%%
% Parameters:
% TopLev    % number of iterations (reconstruction using Pyramid algorithm)
% NrOfVM    % number of vanishing moments of the Daubechies wavelet function
% modelo    % stochastic model used 
% Lm        % number of generated samples
% alpha     % scaling exponent; remember that H = (1+alpha)/2 and d = alpha -0.5 
%           % where 'H' is the Hurst parameter and 'd' is the fractional parameter
%           % of an ARFIMA(p,d,q) model
% p         % shape parameter of the beta probability density function 
%           % In the MWM model, the variance of the multiplier R is given by
%           % var[R] = 1/(2p+1). 
% phi       % AR(1) coefficient
%%%%%%%%%%%%%%%%%%%%%%%%%%%%%%%%%%%%%%%%%%%%%%%%%%%%%%%%%%%%%%%%%%%%%%%%
%
% use as: [Data, nome_arquivo]=Generate(TopLev, NrOfVM, modelo, varargin)
%      or [Data]=Generate(TopLev, NrOfVM, 9, varargin) - modelo 9
% where: 
% for 'modelo'= 1 (FGN, white noise wav. coef.), 'varargin' will be 2 parameters:  "Lm, alpha"
% for 'modelo'= 2 (MWM), 'varargin' sera dois parametros:    "alpha, p"  
% for 'modelo'= 3 (FGN, AR(1) wav. coef.), 'varargin' will be 3 parameters: "Lm, alpha, phi"
% for 'modelo'= 9, 'varargin' will be five parameters:  "Lm, vetor_coefs (C), L, TopLev, wname" 
%
% Example of interactive use: generation of an FGN signal with 4096 samples
%                             alpha = 0.6 (H=[1+alpha]/2=0.8), 
%                             from 1 root scale (approximation) coeficient
%
% >> [Data, archive_name] = Generate(12, 1, 1, 4096,0.6)
% 
% wname =
% db1
% g =
%    0.7071    0.7071
% Lf =
%    2
% h =
%    0.7071   -0.7071
% Data = 
% Columns 1 through 6
%    [1x1 struct]    [1x1 struct]    [1x1 struct]    [1x1 struct]    
%[1x1 struct]    [1x1 struct]
%  Columns 7 through 12
%    [1x1 struct]    [1x1 struct]    [1x1 struct]    [1x1 struct]
%[1x1 struct]    [1x1 struct]
%  Column 13
%    [1x1 struct]
% archive_name =
% fGn_(4096)pts_alpha(0.6)_NrOfVM(1)_TopLev(12)
%
%%%%%%%%%%%%%%%%%%%%%%%%%%%%%%%%%%%%%%%%%%%%%%%%%%%%%%%%%%%%%%%%%%%%%%%%

function [Data, archive_name] = Generate(TopLev, NrOfVM, model, varargin)

N = int2str(NrOfVM); % N is a string
wname = 'db'; % must concatenate N obtain dbN
% Compute the corresponding scaling filter. 
wname = strcat(wname,N)
% Compute the corresponding scaling filter. 
g = dbwavf(wname); 
g = g * sqrt(2)
Lf = length(g) % must be 2N
h = fliplr(g); h = (-1).^(0:Lf-1).*h 
% Eq.(75b) of book "Wavelet Methods for Time Series Analysis"


% call function Model, which generates an approximation on scale j=0.
[Data, archive_name] = Model(TopLev, NrOfVM, h, g, model, varargin{:});
\end{verbatim}
\end{mylisting}

\clearpage
\begin{mylisting}
\begin{verbatim}
function [Data,nome_arquivo] = Model(TopLev, NrOfVM, h, g, modelo, varargin)
	  
    if modelo ~= 2, 
        Lm = varargin{1};
        varargin(1) = []; 
        switch modelo
            case 1, 
                nome_arquivo = ['fGn_(',int2str(Lm),')pts_alpha(',num2str(varargin{1}),')_
                NrOfVM(',int2str(NrOfVM),')_TopLev(',int2str(TopLev),')'] ;
            case 3, 
                nome_arquivo =  ['fGn_(',int2str(Lm),')pts_alpha(',num2str(varargin{1}),')_
                phi(',num2str(varargin{2}),')_NrOfVM(',int2str(NrOfVM),')_
                TopLev(',int2str(TopLev),')'] ;                 
            case 9, 
                 nome_arquivo = [];
        end
        
    else %2 = MWM:
        if NrOfVM ~= 1,
            error('Modelo 2 (MWM) exige que se escolha NrOfVM = 1 para usar a Wavelet de Haar');
        end
        Lm = 2^TopLev; % one root scale coefficient
        
        nome_arquivo = ['MWM_(',int2str(Lm),')pts_alpha(',num2str(varargin{1}),')_p(',
        num2str(varargin{2}),')_TopLev(',int2str(TopLev),')'] ;
    end
    
    Lf = 2*NrOfVM; % filter length
	kn = [0 Lm-1]; 	
	kp = kn; 
	Data{0+(1)}.kp = kn; 
	kc = kn;  
	kd = kn; 
	Data{0+(1)}.kd = kn;
	
	for j=1:TopLev,
        kc = [kd(1)-(Lf-1) kd(2)]; 
        kd = fix (kc/2); 
         
        kn = floor([0 kn(2)-(Lf-1)]/2); 
        kp = floor(kp/2) - [(NrOfVM - 1) 0]; 
        
        Data{j+(1)}.kd = kd;
        Data{j+(1)}.kp = kp; 
        Data{j+(1)}.kn = kn;
	end
	
	Data{TopLev+(1)}.app  = appProcess(diff(kp)+1, modelo, varargin{:}); 
    
    for j=TopLev:-1:1,
        [Data] = detProcess(j,TopLev,Data,modelo,varargin{:});
        Data{j-1+(1)}.app = Recon(j,NrOfVM,Data,h,g); 
    end

    
%%%%%%%%%%%%%%%%%%%%%%%%%%%%%%%%%%%%%%%%%%%%%%%%%%
%internal:
function approxs = appProcess(n, modelo, varargin)
	  switch modelo
        case 1, 
            approxs = 4+randn(1,n);
        case 2, approxs = 4+randn(1,n); 
        case 3, approxs = zeros(1,n); 
        case 9, 
            vetor_coefs = varargin{1};
            L=varargin{2};
            TopLev = varargin{3};
            wname = varargin{4};
            coefsA = appcoef(vetor_coefs,L,wname,TopLev);
            if length(coefsA) == n ,
                approxs = coefsA; 
            else
                warning('Numero de coeficientes da dwt nao bate com o 
                necessario para reconstituir o sinal');
                approxs = coefsA(1:n);
            end
        otherwise, disp('Modelo nao implementado'); approxs = []; 
    end

            
%%%%%%%%%%%%%%%%%%%%%%%%%%%%%%%%%%%%%%%%%%%%%%%%%%%%%%%%%%%%%%%%%%%
%internal:
function [Data] = detProcess(level, TopLev, Data, modelo, varargin)
    if modelo ~= 9,
        alpha = varargin{1}; 
        ctrvariance = 2^( (level - TopLev) * alpha/2 );
    end
    
    n = diff( Data{level+(1)}.kp ) + 1;
    
    switch modelo
        case 1, 
            Data{level+(1)}.det = ctrvariance*randn(1,n); 
        case 2, %MWM:
            p = varargin{2};
            if (level == TopLev) & (p == 0),
                p = (2^alpha - 1)/(2 - 2^alpha);
                Data{0+(1)}.p = p;
                for i = 1:TopLev,
                    p = (2*p+1)/2^alpha - 1;
                    Data{i+(1)}.p = p;
                end
            elseif (level == TopLev) & (p > 0),
                Data{TopLev+(1)}.p = p;
                for i = TopLev-1:-1:0,
                    p = (2^alpha*(p+1)-1)/2;
                    Data{i+(1)}.p = p;
                end
                p = Data{TopLev+(1)}.p;
            elseif level ~= TopLev,
                p = Data{level+(1)}.p;
            end
            R01 = betarnd(p,p,1,n);
            R = 2.*R01-1;
            Data{level+(1)}.det = R.*Data{level+(1)}.app;
        case 3, phi = varargin{2}; 
            inject = ctrvariance*randn(1,n);
            Data{level+(1)}.det(1) = phi*0 + inject(1);
            for i=2:n,
                Data{level+(1)}.det(i) = phi*Data{level+(1)}.det(i-1) + inject(i);
            end
            clear inject;
        case 9, 
            vetor_coefs = varargin{1};
            L = varargin{2};
            coefsD = detcoef(vetor_coefs,L,level); 
            if length(coefsD) == n ,
                Data{level+(1)}.det = coefsD; 
            else
                warning('Numero de coeficientes da dwt nao bate com o 
                necessario para reconstituir o sinal');
                Data{level+(1)}.det = coefsD(1:n);
            end     
            
        otherwise, error(['Modelo ', int2str(modelo), ' nao implementado!!!']); 
    end
\end{verbatim}
\end{mylisting}

\clearpage
\begin{mylisting}
\begin{verbatim}

function approxs = Recon(level, NrOfVM, Data, h, g) % h will be high-pass filter coefficients.
    Lf = length(h);
    
    kp = Data{level + (1)}.kp; 
    
    % Eqs. (7.11), (7.12), (7.13) and (7.14) of Backar M.Sc thesis (see  p.49):
    %%%%%%%%%%%%%%%%%%
    
    ku = 2*kp; 
    
    kc = ku + [0 Lf-1]; 
    
    %%%%%%%%%%%%%%%%%%%
    
    kp = Data{level-1 + (1)}.kp; 
    
    appro =  Data{level + (1)}.app;
    detail =  Data{level + (1)}.det;
    
    appup = dyadup(appro,0);
    detup = dyadup(detail,0);
    
    appconv = conv(appup,g); 
    detconv = conv(detup,h); 
    
    indices = [kp(1)-kc(1)+(1):kp(2)-kc(1)+(1)];
    
    appro = appconv(indices) + detconv(indices);
    
    approxs = appro;
\end{verbatim}
\end{mylisting}
