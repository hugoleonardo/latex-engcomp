\chapter{C�digo em S para S-PLUS}
\label{apend-cod-S}

\begin{mylisting}
\begin{verbatim}
########################################################################
# author: Alexandre B. de Lima                                                 
# version of: 24/12/2007                                                       
#                                                                              																			 	
# Given a long memory time series, this script performs                        
# - statistical tests for LRD                                                  
# - estimation of the Hurst parameter                                          
# - estimation of an FARIMA(p,d,q), a high order AR(p) (using the AIC 
# criterion), and an AR(2) model #  
# - "k-steps-ahead-forecasts" computed from the Kalman filter prediction 
# equations associated to a state-space representation of the AR(p) 
# model 
#                                                  
# Copyright(c) 2007 Universidade de S�o Paulo, Laborat�rio de 
# Comunica��es e Sinais
# Permission is granted for use and non-profit distribution providing that
# this notice be clearly maintained. The right to distribute any portion  
# for profit or as part of any commercial product is specifically reserved
# for the author.                                                              
##########################################################################

setTextOutputRouting(normalTextWindow = "Report", errorWindow = "Report")

# Input time series
namets=scan(file="",what=character(),n=1)           
# Enter the name of the ASCII file which contains the time series 
# to be analyzed. Note that this file must be in your S+ home directory

myts.df = importData(file=namets)                   
# Load time series from the specified file

myts.matrix = as.matrix.data.frame(myts.df)

plot(myts.matrix, type="l")
title("time series")
axes(xlab="t", ylab="") 
guiSave("GraphSheet", Name="GSD2",FileName=paste("myts_plot",".sgr",
sep=""))
guiClose("GraphSheet","GSD2")

# histogram of time series
hist(myts.matrix)
title("Histogram of time series")
guiSave("GraphSheet", Name="GSD2",FileName=paste("myts_hist",".sgr",
sep=""))
guiClose("GraphSheet","GSD2")

# QQ-plot of time series
qq.plot.myts <- qqnorm(myts.matrix)
title("qqplot of time series")
guiSave("GraphSheet", Name="GSD2",FileName=paste("myts_qqplot",".sgr",
sep=""))
guiClose("GraphSheet","GSD2")

mu = mean(myts.matrix)
# Be careful here ... remember that the coefficients in the function "ar" 
# are for the series with the mean(s) removed. You must also demean the 
# series in order to have a meaningful state-space representation
# when using the "GetSsfArma" function (try not to use and you will see 
# for yourseld that the Kalman recursions do not work)

# ---------------------------------------------------------------------
# Testing for stationarity (null hypothesis that series is I(0), see 
# pp. 123 of "Modeling Financial Time Series with S-PLUS") 
kpss.out = stationaryTest(myts.matrix, trend="c")
kpss.out

# demean the series
myts.matrix = myts.matrix - mu 

N = length(myts.matrix)    # number of points of time series
h = 100                    # t(origin of forecasts) = N-h
myts = timeSeries(myts.df) # create a "timeSeries" object

# Estimate power spectrum density of demeaned time series
spec.pgram(myts.matrix, demean=T, spans=c(9,7), plot=T) 
# smoothed periodogram, Daniell smoother
guiSave("GraphSheet", Name="GSD2",FileName=paste("pgram_myts",".sgr",
sep=""))
guiClose("GraphSheet","GSD2")
spec.pgram(myts.matrix, demean=T, spans=1, plot=T) 
guiSave("GraphSheet", Name="GSD2",
FileName=paste("raw_pgram_myts",".sgr",sep=""))
guiClose("GraphSheet","GSD2")

# ----------------------------------
# Statistical Tests for Long Memory

# R/S statistic
rosTest(myts.matrix)

# GPH test
#gph.myts = gphTest(myts.matrix,demean=T, spans=c(9,7))
gph.myts = gphTest(myts.matrix,taper=0.1, demean=F, spans=c(9,7))
gph.myts

# -----------------------------
# Estimation of Hurst Parameter 

# whittle's method
d.w = d.whittle(myts.matrix, demean= F, output="d") 
d.w

# periodogram method
d.smooth.pgram = d.pgram(myts.matrix,taper=0.1, demean=F, spans=c(9,7),
output="d")
d.smooth.pgram

# R/S Analysis
d.ros(myts.matrix,minK=50,k.ratio=2,minNumPoints=10,output="d",plot=T)
# because the length of time series is sufficiently large (N=4096, 
# for instance)
guiSave("GraphSheet", Name="GSD2",FileName=paste("d_est_ros",".sgr",
sep=""))
guiClose("GraphSheet","GSD2")

#d.ros(myts.matrix,minK=4,k.ratio=2,minNumPoints=10,output="d",plot=T) 
# because it's a small time series, like the Nile river minima 
# for the years 1007 to 1206 
#guiSave("GraphSheet", Name="GSD2",FileName=paste("d_est_ros",".sgr",
sep=""))
#guiClose("GraphSheet","GSD2")

# -------------------------------------------------------------------
# Estimate FARIMA(p,d,q) model and SEMIFARmodel 

myts.farima.fit.bic = FARIMA(myts.df,p.range=c(0,2),q.range=c(0,2),
mmax=1)
# estimated model
myts.farima.fit.bic$model  
# unit roots ?
myts.farima.fit.bic$m      
m = myts.farima.fit.bic$m

myts.semifar.fit = SEMIFAR(myts.df,p.range=c(0,2),trace=F)
myts.semifar.fit$model
myts.semifar.fit$m

# -----------------------------------------------------------------
# plot sample ACF of series and theoretical ACF of the fitted model  
if  (m == 0) 
{
	# diagnostic of the fitted FARIMA model
	plot(myts.farima.fit.bic)                          
	guiSave("GraphSheet", Name="GSD2",
	FileName=paste("diagnostics_farima_fit",".sgr",sep=""))
	guiClose("GraphSheet","GSD2")

	# plot periodogram of FARIMA residuals
	psd.resid.farima.fit <- spec.pgram(myts.farima.fit.bic$residuals, 
	spans=c(9,7), plot=T) 
	guiSave("GraphSheet", Name="GSD2",FileName=paste
	("pgram_residuals_farima_fit",".sgr",sep=""))
	guiClose("GraphSheet","GSD2")

	# plot theoretical ACF of estimated FARIMA model
	farima.mod  = list(ar=myts.farima.fit.bic$model$ar, 
	ma=myts.farima.fit.bic$model$ma, sigma2=1, 
	d=myts.farima.fit.bic$model$d)
	arfima.acf  = acf.FARIMA(farima.mod,lag.max=200)
	plot(arfima.acf$lags, arfima.acf$acf/arfima.acf$acf[1], 
	type="h",xlab="lags", ylab="ACF")
	guiSave("GraphSheet", Name="GSD2",FileName=paste("acf_FARIMA",
	".sgr",sep=""))
  guiClose("GraphSheet","GSD2")

	# ---------------------------------
	# estimate an AR(p) model using AIC
	myts.ar.fit.aic = ar(myts, aic=T, order.max=40) 
	myts.ar.fit.aic$order
	myts.ar.fit.aic$ar
	coef.ar.fit.aic = as.matrix(myts.ar.fit.aic$ar)
	exportData(coef.ar.fit.aic, "coef_ar_fit_aic.txt",type="ASCII",
	colNames=FALSE)

	# ---------------------------------
	# plot SACF of signal, theoretical ACFs of estimated FARIMA and 
	# high-order AR models 
	sacf.myts   = acf(myts.matrix,lag=200)
	lines(arfima.acf$lags, arfima.acf$acf/arfima.acf$acf[1])
	arp.mod = list(ar=as.vector(myts.ar.fit.aic$ar),sigma2=1,d=0)
	ar.acf = acf.FARIMA(arp.mod,lag.max=200)
	lines(ar.acf$lags,ar.acf$acf/ar.acf$acf[1])
	guiSave("GraphSheet", Name="GSD2",FileName=paste("myts_SACF_ACF",
	".sgr",	sep=""))
	guiClose("GraphSheet","GSD2")
	
	# ----------------------------------
	# diagnostics of the AR(p) model fit
	resid.ar.aic.fit = myts.ar.fit.aic$resid
	acf.resid.ar.aic.fit= acf(resid.ar.aic.fit, type="correlation", plot=T)
	guiSave("GraphSheet", Name="GSD2",FileName=paste("resid_ar_AIC_SACF",
	".sgr",sep=""))
	guiClose("GraphSheet","GSD2")
	spec.pgram(resid.ar.aic.fit, demean=T, spans=c(9,7), plot=T) 
	guiSave("GraphSheet", Name="GSD2",FileName=paste("pgram_resid_ar_AIC",
	".sgr",sep=""))
	guiClose("GraphSheet","GSD2")
	
}
else
{
	# series has unit roots!
	# diagnostic of the fitted SEMIFAR model
	plot(myts.semifar.fit)                          
	guiSave("GraphSheet", Name="GSD2",
	FileName=paste("diagnostics_semifar_fit",".sgr",sep=""))
	guiClose("GraphSheet","GSD2") 
	
	# plot periodogram of SEMIFAR residuals
	psd.resid.semifar.fit <- spec.pgram(myts.semifar.fit$residuals, 
	spans=c(9,7), plot=T) 
	guiSave("GraphSheet", Name="GSD2",FileName=
	paste("pgram_residuals_semifar_fit",".sgr",sep=""))
	guiClose("GraphSheet","GSD2")

	# plot theoretical ACF of estimated FARIMA model (here I use 
	# the Whittle estimate for "d")
	farima.mod  = list(ar=myts.farima.fit.bic$model$ar, 
	ma=myts.farima.fit.bic$model$ma, sigma2=1, d=d.w)
	arfima.acf  = acf.FARIMA(farima.mod,lag.max=200)
	plot(arfima.acf$lags, arfima.acf$acf/arfima.acf$acf[1], type="h",
	xlab="lags", ylab="ACF")
	guiSave("GraphSheet", Name="GSD2",FileName=paste("acf_FARIMA",
	".sgr",sep=""))
  guiClose("GraphSheet","GSD2")

	#sacf.myts  = acf(myts.matrix,lag=200)
	#guiSave("GraphSheet", Name="GSD2",FileName=paste("myts_SACF",
	".sgr",sep=""))
	#guiClose("GraphSheet","GSD2")
	
	# ---------------------------------
	# estimate an AR(p) model using AIC
	myts.ar.fit.aic = ar(myts, aic=T, order.max=40) 
	myts.ar.fit.aic$order
	myts.ar.fit.aic$ar
	coef.ar.fit.aic = as.matrix(myts.ar.fit.aic$ar)
	exportData(coef.ar.fit.aic, "coef_ar_fit_aic.txt",type="ASCII",
	colNames=FALSE)
	
	# ---------------------------------
	# plot SACF of signal, theoretical ACFs of estimated FARIMA and high-order 
	# AR models 
	sacf.myts   = acf(myts.matrix,lag=200)
	lines(arfima.acf$lags, arfima.acf$acf/arfima.acf$acf[1])
	arp.mod = list(ar=as.vector(myts.ar.fit.aic$ar),sigma2=1,d=0)
	ar.acf = acf.FARIMA(arp.mod,lag.max=200)
	lines(ar.acf$lags,ar.acf$acf/ar.acf$acf[1])
	guiSave("GraphSheet", Name="GSD2",FileName=paste("myts_SACF_ACF",
	".sgr",sep=""))
	guiClose("GraphSheet","GSD2")

	# ----------------------------------
	# diagnostics of the AR(p) model fit
   resid.ar.aic.fit = myts.ar.fit.aic$resid
   acf.resid.ar.aic.fit= acf(resid.ar.aic.fit, type="correlation", plot=T)
   guiSave("GraphSheet", Name="GSD2",FileName=paste("resid_ar_AIC_SACF",
   ".sgr", sep=""))
   guiClose("GraphSheet","GSD2")
   spec.pgram(resid.ar.aic.fit, demean=T, spans=c(9,7), plot=T) 
   # smoothed periodogram, Daniell smoother
   guiSave("GraphSheet", Name="GSD2",FileName=
   paste("pgram_resid_ar_AIC",".sgr", sep=""))
   guiClose("GraphSheet","GSD2")
	
}

myts.ar.fit.aic$order
	
# -----------------------
# estimate an AR(2) model
myts.ar2.fit = ar(myts, aic=F, order=2) 
myts.ar2.fit$order
myts.ar2.fit$ar
coef.ar2.fit = as.matrix(myts.ar2.fit$ar)
exportData(coef.ar2.fit, "coef_ar2_fit.txt",type="ASCII",colNames=FALSE)

# diagnostics of the AR(2) model fit
resid.ar2.fit = myts.ar2.fit$resid
acf.resid.ar2.fit= acf(resid.ar2.fit, type="correlation", plot=T)
guiSave("GraphSheet", Name="GSD2",FileName=paste("resid_ar2_SACF",".sgr",
sep=""))
guiClose("GraphSheet","GSD2")
spec.pgram(resid.ar2.fit, demean=T, spans=c(9,7), plot=T) 
# smoothed periodogram, Daniell smoother
guiSave("GraphSheet", Name="GSD2",FileName=paste("pgram_resid_ar2",".sgr",
sep=""))
guiClose("GraphSheet","GSD2")


# --------------------------------
# Forecasts with the Kalman Filter
# AR(p) model using AIC
ssf.myts.ar.fit.aic = GetSsfArma(ar=myts.ar.fit.aic$ar, ma=NULL, 
sigma=sqrt(myts.ar.fit.aic$var.pred))
myts.new = c(myts.matrix[1:(N-h)],rep(NA,h)) 
# you must append a seq of h missing values 
# (see p.532 of the book "Modeling Financial Time Series with S-PLUS")

Kalman.est = SsfMomentEst(myts.new, ssf.myts.ar.fit.aic, task="STPRED")
# you must use a demeaned time series!

myts.fcst  = Kalman.est$response.moment + mu
fcst.var   = Kalman.est$response.variance
exportData(myts.fcst[(N-h+1):N], "fcst_Kalman_ar_aic.txt",type="ASCII",
colNames=FALSE)
upper      =  myts.fcst + 1.96*sqrt(fcst.var)
lower      =  myts.fcst - 1.96*sqrt(fcst.var)
upper[1:(N-h)] = lower[1:(N-h)] = NA 
exportData(upper[(N-h+1):N], "upper_Kalman_ar_aic.txt",type="ASCII",
colNames=FALSE)
exportData(lower[(N-h+1):N], "lower_Kalman_ar_aic.txt",type="ASCII",
colNames=FALSE)

#tsplot((myts.new[(N-2*h+1):N]+mu),myts.fcst[(N-2*h+1):N],
upper[(N-2*h+1):N], lower[(N-2*h+1):N], lty=c(1,2,2,2))

tsplot((myts.matrix[(N-2*h+1):N]+mu),myts.fcst[(N-2*h+1):N],
upper[(N-2*h+1):N], lower[(N-2*h+1):N], lty=c(1,2,2,2))
guiSave("GraphSheet", Name="GSD2",FileName=
paste("myts_KalmanPred_ARaic",".sgr",sep=""))
guiClose("GraphSheet","GSD2")

# --------------------------------
# Forecasts with the Kalman Filter
# AR(2) model
ssf.myts.ar2.fit = GetSsfArma(ar=myts.ar2.fit$ar, ma=NULL, 
sigma=sqrt(myts.ar2.fit$var.pred))
Kalman.est2 = SsfMomentEst(myts.new, ssf.myts.ar2.fit, task="STPRED")
myts.fcst2  = Kalman.est2$response.moment + mu
fcst.var2   = Kalman.est2$response.variance
exportData(myts.fcst2[(N-h+1):N], "fcst_Kalman_ar2.txt",type="ASCII",
colNames=FALSE)
upper2      =  myts.fcst2 + 1.96*sqrt(fcst.var2)
lower2      =  myts.fcst2 - 1.96*sqrt(fcst.var2)
upper2[1:(N-h)] = lower2[1:(N-h)] = NA 
exportData(upper2[(N-h+1):N], "upper2_Kalman_ar_aic.txt",type="ASCII",
colNames=FALSE)
exportData(lower2[(N-h+1):N], "lower2_Kalman_ar_aic.txt",type="ASCII",
colNames=FALSE)

tsplot((myts.matrix[(N-2*h+1):N]+mu),myts.fcst2[(N-2*h+1):N],
upper2[(N-2*h+1):N],
lower2[(N-2*h+1):N], lty=c(1,2,2,2))
guiSave("GraphSheet", Name="GSD2",FileName=paste("myts_KalmanPred_AR2",
".sgr",sep=""))
guiClose("GraphSheet","GSD2")

######################################################################

setTextOutputRouting(normalTextWindow = "Default", errorWindow 
= "Default")
\end{verbatim}
\end{mylisting}