\documentclass[pdf,rico,slideColor,colorBG]{prosper}

\usepackage[brazil]{babel}
\usepackage[latin1]{inputenc}
\usepackage{pstricks,pst-node,pst-text,pst-3d}
\usepackage{fancyvrb}
\usepackage{graphicx}
\usepackage{amsmath}
% Definition of new colors
\newrgbcolor{LemonChiffon}{1. 0.98 0.8}
\newrgbcolor{LightBlue}{0.68 0.85 0.9}
\title{Usando \LaTeX\ e Prosper para Apresenta��es}
\subtitle{uma breve introdu��o}
\author{{\green Thadeu Penna}}
\institution{%
  Instituto de F�sica\\
  UFF}

\newcommand{\setas}{\hfill
\Acrobatmenu{PrevPage}{$\Leftarrow$}\Acrobatmenu{NextPage}{$\Rightarrow$}}

\begin{document}
\maketitle

%---------------------------------------------------------------------- SLIDE -
\overlays{6}{

\begin{slide}{Vantagens do \texttt{Prosper}}
\hypertarget{SECOND}{Segundo Slide}
\begin{itemstep}
\item Aproveitar trechos de artigos batidos em \LaTeX\
\item Mais f�cil escrever f�rmulas 
\item Preocupa��o com o que vai ser apresentado
\item Mais f�cil fazer conex�es
\item � mais port�til: arquivo pdf
\item F�cil criar transpar�ncias pra emerg�ncias
\end{itemstep}
\end{slide}
}
%------------------------------------------------------------------------------

%---------------------------------------------------------------------- SLIDE -
\begin{slide}{Estrutura do Arquivo \texttt{Prosper}}
  \begin{center}
    \includegraphics[height=\textheight]{prosper-structure.eps}
  \end{center}
\end{slide}

%------------------------------------------------------------------------------

%---------------------------------------------------------------------- SLIDE -
\begin{SaveVerbatim}{VerbEnv}
\begin{slide}[transi��o]{t�tulo} 
\end{SaveVerbatim}

\overlays{7}{%
\begin{slide}{Transi��es}
\texttt{Prosper} tem sete tipos de transi��es

\BUseVerbatim{VerbEnv}

\begin{itemstep}
\item Split;
\item Blinds;
\item Box;
\item Wipe;
\item Dissolve;
\item Glitter;
\item Replace.
\end{itemstep}
\end{slide}
}
%------------------------------------------------------------------------------


%---------------------------------------------------------------------- SLIDE -

\overlays{6}{%
\begin{slide}[Split]{Estilo}
\texttt{Prosper} tem 23 estilos pr�-definidos

\begin{tabular}{ccc}
\begin{minipage}[c]{0.3\textwidth}
\includegraphics[width=0.9\textwidth]{alienglow.eps}
\end{minipage}
&
\FromSlide{2}
\begin{minipage}[c]{0.3\textwidth}
\includegraphics[width=0.9\textwidth]{azure.eps}
\end{minipage}
&
\FromSlide{3}
\begin{minipage}[c]{0.3\textwidth}
\includegraphics[width=0.9\textwidth]{rico.eps}
\end{minipage}\cr
\FromSlide{4}
\begin{minipage}[c]{0.3\textwidth}
\includegraphics[width=0.9\textwidth]{prettybox.eps}
\end{minipage}
&
\FromSlide{5}
\begin{minipage}[c]{0.3\textwidth}
\includegraphics[width=0.9\textwidth]{frames.eps}
\end{minipage}
&
\FromSlide{6}
\begin{minipage}[c]{0.3\textwidth}
\includegraphics[width=0.9\textwidth]{blends.eps}
\end{minipage}

\end{tabular}

\end{slide}
}

%------------------------------------------------------------------------------

\begin{SaveVerbatim}{Figura}  
  \begin{center}
    \includegraphics[height=\textheight]{.eps}
  \end{center}
\end{SaveVerbatim}

\overlays{4}{%
\begin{slide}[Box]{Figuras}

\begin{itemstep}
\item Figuras devem estar no formato EPS, como no \LaTeX .
\item Jpegs podem ser convertidos com \texttt{jpeg2ps} ou \texttt{convert}
\item Recomend�vel o uso do pacote \texttt{graphicx}
\item {\scriptsize \BUseVerbatim{Figura}}
\end{itemstep}
\end{slide}
}
%---------------------------------------------------------------------- SLIDE -

\begin{SaveVerbatim}{minipage}

\begin{tabular}{cc}
\begin{minipage}[c]{\textwidth}
\includegraphics[width=0.9\textwidth]{.eps}
\end{minipage}
&
\begin{minipage}[c]{\textwidth}
\includegraphics[width=0.9\textwidth]{.eps}
\end{minipage}
\end{tabular}

\end{SaveVerbatim}

\overlays{2}{
\begin{slide}{Figuras lado a lado}

\begin{itemstep}
\item Figuras lado a lado usam \texttt{minipage}
\item {\scriptsize \BUseVerbatim{minipage}}
\end{itemstep}

\end{slide}
}
%------------------------------------------------------------------------------

\begin{SaveVerbatim}{overlays}
\overlays{n}{
\begin{slide}
...
\end{slide}
}
\end{SaveVerbatim}

\overlays{5}{
\begin{slide}{Overlays}
\begin{itemstep}
\item Cada slide � uma superposi��o de ``overlays��
\item Anima��es s�o feitas com apresenta��es das overlays.
\item {\scriptsize \BUseVerbatim{overlays}}
\item \texttt{\\fromSlide\{p\}\{texto\}} ou \texttt{\\untilSlide\{p\}\{texto\}}
\item \texttt{\\FromSlide\{p\}} ou \texttt{\\UntilSlide\{p\}}
\end{itemstep}

\end{slide}
}
%---------------------------------------------------------------------- SLIDE -

\begin{SaveVerbatim}{itemsteps}
\begin{itemstep}
\item ..
\item ...
\item ....
\end{itemstep}
\end{SaveVerbatim}

\overlays{2}{
\begin{slide}{ItemSteps}
\begin{itemstep}
\item F�cil fazer com \texttt{itemstep}
\item {\scriptsize \BUseVerbatim{itemsteps}}
\end{itemstep}
\end{slide}
}

%--------------------------------------------------------------------- SLIDE - 

\begin{slide}[Dissolve]{Tri�ngulo de Landau \setas}

Usamos comando do Acrobat Menu e PSTricks

\psset{linecolor=yellow,linewidth=2pt}
\rput(3,-4){
\psline{->}(0.5,0.5)(1.75,2.5)
\psline{->}(2,2.5)(0.75,0.5)
\psline{->}(3.75,0.5)(2.25,2.5)
\psline{->}(2.5,2.5)(4.0,0.5)
\psline{->}(0.5,0)(4,0)
\psline{->}(4,0.25)(0.5,0.25)
\rput(2,3){\psframebox[linewidth=0pt]{\yellow EXPERI�NCIA}}
\rput(-0.6,0.1){\psframebox[linewidth=0pt]{\yellow TEORIA}}
\rput(5.6,0.1){\psframebox[linewidth=0pt]{\yellow SIMULA��O}}
}
\end{slide}

%---------------------------------------------------------------------- SLIDE -
\overlays{2}{%
\begin{slide}{Diagramas}
Diagramas usando \LaTeX.
\onlySlide{2}{%
  Se o diagrama  e textos est�o no mesmo n�vel 
  � f�cil adicionar links entre \rnode{LIEN}{eles}.}%

\vspace{0.4cm}
{\tiny
\begin{equation*}
\setlength{\arraycolsep}{1cm}
\def\tX{\tilde{\tilde{X}}}
\begin{array}{cc}
        (X-A,N-A)\rnode{a}{} & \rnode{b}{}(\tX,a)\\[1.5cm]
        (X,N)\rnode{c}{} & \rnode{d}{}(\tX,N)\\[1.5cm]
\end{array}
\psset{nodesep=5pt,arrows=->}
\onlySlide*{2}{\nccurve[linecolor=white,angleA=270,angleB=180]{LIEN}{d}}%
\ncline[linecolor=white]{a}{b}\Aput{a}
\ncline[linecolor=white]{a}{c}\Bput{r}
\ncline[linecolor=white,linestyle=dashed]{c}{d}\Bput{b}
\ncline[linecolor=white]{b}{d}\Bput{s}
\end{equation*}}
\end{slide}
}

%------------------------------------------------------------------------------


%---------------------------------------------------------------------- SLIDE -
\begin{slide}[Dissolve]{Texto em Curvas}

\vspace{3cm}
\pstextpath{\psccurve[linestyle=none](.5,0)(3.5,1)(3.5,0)(.5,1)}{\green E muitos outros efeitos\dots}
\end{slide}
%------------------------------------------------------------------------------


%---------------------------------------------------------------------- SLIDE -
\overlays{3}{%
\begin{slide}{Householder formula}
\small
The Householder formula below lets you compute $f^{-1}(x)$ for an arbitrary
$f$.
{\scriptsize
\begin{equation}\label{Householder}
x_{k+1}\mapsto \Phi_n(x_k)=x_k+(n-1)
\frac{\bigl(\frac{1}{f(x_k)}\bigr)^{n-2}}{\bigl(\frac{1}{f(x_k)}\bigr)^{n-1}}+
f(x_k)^{n+1}%
\fromSlide*{2}{\rnode{NA}{\pscirclebox[linecolor=red]{\psi}}}
\onlySlide*{1}{\rnode{NA}{\pscirclebox[linecolor=red,linestyle=none]{\psi}}}
\end{equation}}

\FromSlide{2}%
where $n\geq 2$ and \rnode{NB}{$\psi$} is an arbitrary function.
\fromSlide*{3}{\nccurve[linecolor=red,angleA=90,angleB=270]{->}{NB}{NA}}

\OnlySlide{3}%
Formula~\eqref{Householder} gives an iteration of order $n$ converging
towards $x_*$ such that: $f(x_*)=0$.
\end{slide}
}
%------------------------------------------------------------------------------

\begin{slide}{Compila��o}

  \begin{center}
    \includegraphics[width=\textwidth]{compilation.eps}
  \end{center}
\end{slide}  
  

\begin{slide}{Links}
\begin{verbatim}
\href{run:seal.mp3}
\end{verbatim}
\href{run:seal.mp3}{Clique aqui}

\end{slide}

%---------------------------------------------------------------------- SLIDE -
\begin{slide}{Last slide}
  This is the \hypertarget{LAST}{last} slide. Do you want to go to the 
  \hyperlink{SECOND}{{\green second one}}? 
  
  Uso de \texttt{hyperlink} e \texttt{hypertarget}.  
\end{slide}
%---------------------------------------------------------------------- SLIDE -


\end{document}

%%% Local Variables: 
%%% mode: latex
%%% TeX-master: t
%%% End: 
